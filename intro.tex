\section{Introduction}

\begin{itemize}

    \item What is the subject of the study? Describe the
        economic/econometric problem.

    \item What is the purpose of the study (working hypothesis)?

    \item What do we already know about the subject (literature
        review)? Use citations: {\it \citet{Gallant:87} shows that...
        Alternative Forms of the Wald test are considered
        \citep{Breusch&Schmidt:88}.}

    \item What is the innovation of the study?

    \item Provide an overview of your results.


    \item Outline of the paper:\\
        {\it The paper is organized as follows. The next section describes the
        model under investigation. Section \ref{Sec:Literature Review} reviews the up-to-date research works.
        and Section \ref{Sec:Results} presents the results. Finally, Section
        \ref{Sec:Conc} concludes.}

    \item The introduction should not be longer than 4 pages.

\end{itemize}

\par More and more the usage of RDF is increasing in many fields in computer science. RDF data representation helps in supporting the machine to perfrom the normally manual computation work in an automatic fashion. Moreover, the machines will be smarter to understand the data which is represented in RDF format. 
\vspace{5mm} %5mm vertical space


The quality of RDF data needs to be ensure before proceeding of any further processing. Most of current parsers which focus on detection of the syntax error fail to detect more than one error, especially, of RDF data represented in Turtle or NTriples format. 

\vspace{5mm} %5mm vertical space

This study was encouraged by the tremendous data representation of either Turtle and NTriples. Hence, the intention of the study to afford a user-friendly syntax checker or parser. Such parser or syntax checker should give all errors can be detect inside such data.

\vspace{5mm} %5mm vertical space


The thesis is organized as follows. The next section presents the
problem description. Section \ref{Sec:Review} reviews the up-to-date research works. Section \ref{Sec:Data} describes the data set
and Section \ref{Sec:Results} presents the results. Finally, Section
\ref{Sec:Conc} concludes