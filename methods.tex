\section{Method/Model/Theory}\label{Sec:Method}

\begin{itemize}

    \item How was the data analyzed ?

    \item Present the underlying economic model/theory and
        give reasons why it is suitable to answer the given problem.

    \item Present econometric/statistical estimation method and
        give reasons why it is suitable to answer the given problem.

    \item Allows the reader to judge the validity of the study and
        its findings.

    \item Depending on the topic this section can also be split up
        into separate sections.

\end{itemize}
The collection of the data to be used for the evaluation of the suggested approach were collected from different site. The random selection of data helps to reduce the introduced bias.

The coming days produce more and more data represented in different format. The focus of this study is on data of RDF triples. Before any further processing of data, the data quality must be ensured and verified. If no such verification is performed, it can leads to misleading result which affects decision taken by decision-makers, either or human of machines. 

Several tools were founded in regards to checking syntax errors in RDF data, few of them as was mentioned in \ref{Sec:Review} proposed a formal procedure to find one than one syntax error rather than implement such procedure in their tools. The result of this study not only provide a formal process of searching syntactic errors of RDF code, rather  
