\chapter{Conclusions and Future Work}
\label{ch:conclusions}

\section{Conclusions}

This thesis presents RDF-Doctor, an approach for fault-tolerant and automatic error detection and recovery for RDF data.
For error detection, RDF-Doctor primarily relies on a predefined set of grammar rules which cover a comprehensive list of syntactic errors occurring commonly in RDF data.
The ANTLR framework is used to generate the parser based on the given grammar.
For each of the detected errors, RDF-Doctor provides friendly and precise error messages helping users to easily locate and correct them.
Furthermore, RDF-Doctor using an internal mechanism for automatic error correction is able to recover a subset of syntax errors without user intervention.

We performed three different empirical evaluations to asses the effectiveness and efficiency of RDF-Doctor on various scenarios. Such experiments include 1) a Suite of Tests predefined by W3C Consortium; 2) a synthetic experiment simulating a working day of an ontology engineer where the number and the type of syntax errors are randomly generated and 3) increasing the size of ontologies being parsed and the number of syntax errors.
The achieved results provide evidence that the effectiveness of RDF-Doctor is higher for each error that is modeled in the grammar. 
Furthermore, the efficiency is not impacted by the number and the type of errors, but from the size of the ontology which is expended since  an increasing number of triples to be parsed RDF-Doctor consumes more time. 


%This chapter concludes this study. Mainly, the focus was to detect all syntax errors in RDF input. To achieve the referred goal, The ANTLR tool as a parser generator was used. As a use case for this study, RDF serialization formats of both Turtle and NTriples were investigated, where the grammar that describes there syntax was invented.

%In addition, to detect the syntax errors, error production rules were specified in the grammar. In case a sequence of input matches one of such rule, the parser will rise an error occurrence notification, then the error listeners class will save this error to a list of syntax errors. This list of errors later can help to recover some of these errors if it has one straight forward error which has a well-know syntax format. 

%According to our research in the related work in Chapter \ref{ch:related}, it can be clearly seen that there is a shortage of tools which list all the syntax errors in RDF input. This shortage is found in both RDF editors to create, modify RDF text and as well, in plug-ins which dealing with RDF syntax checking. To fill the gap, this study starts with searching the available possibilities to achieve the goal. Thanks for ANTLR to help us achieving the objectives by provides a grammar which can hold error production rules. 

%As a use case, Turtle, one of RDF serialization formats, was used in this study. The grammar designed based on the syntax of turtle. However, it is easy to replace with another RDF serialization formats to achieve the same objective. Additionally, the grammar can be changed to generate a parser in a different programming language with the help of ANTLR framework.   

\section{Limitations}
%During the implementation of RDF-Doctor, an approach of injecting error production rules inside the grammar which the internal parser of RDF-Doctor built based on. 
Although RDF-Doctor is able to detect different types of syntax errors and correct a subset of them, a number of limitations are still present. 
%These limitations are discussed in the following text and it can be considered in future work:  
\begin{itemize}

    \item \textbf{Supporting only Turtle or N-Triples:} RDF-Doctor is limited to parse only those RDF serialized either in Turtle or N-Triples formats. 
    %There are use as use cases to prove the concept. 
    \item \textbf{Restriction in automatic error correction:} Only a subset of  detected syntax errors can be resolved. 
    The approach is limited only to simple syntax errors which are commonly performed by ontology engineers, such as missing a dot at the end of a triple, adding more than one dot at the end of a triple, or terminating a triple end with a semicolon instead of a dot.   

    \item \textbf{Releasing a high number of false positives:} When particular syntax errors are unrecognized, the internal parser of RDF-Doctor tries to handle this situation by deleting, inserting, or replacing the unrecognized sequence of tokens until it can match them with certain rules.
    This usually leads to false positives due to the high probability of matching these tokens with incorrect rules.
    The next following tokens are considered as syntax errors despite being correct syntax forms which causes the increase of false positives.  

\end{itemize}


\section{Future Work}

%The Semantic Web field is rapidly researched, engineered, and developed and there will be a need for new tools and modern plugins that can be developed in different programming languages. 
%Moreover, the diverse RDF serializations have the same shortage of fault-tolerant parsers and automatic error recovery tools. 
%Equally important a very wide usage of the field and being an open research, opens the road for a couple of new ideas and advanced approaches which will shape the future of the semantic web. All of what was mentioned motivates the future work continuation in the same field. 
There exists a number of potential directions in which our approach can be extended and advanced. 
In the following, these directions are discussed more in detail:   

\subsection{Error Detection Improvement}
In order to improve syntax error detection of RDF data, the next version of RDF-Doctor should able to recognized all types of syntax errors. 
This improvement can done by
%a new version of ANTLR framework that could recognized the syntax more precisely, with the grammar rules, and since the development of ANTLR framework has an active status that can be taken into consideration; 
developing a new module which handles those types of errors and skips them until the next recognized tokens are found and identified. 

\subsection{Comprehensive Error Recovery}

%Instead of limiting RDF-Doctor error recovery only to simple syntax errors, a comprehensive error recovery is planned. RDF-Doctor should able to resolved as much as possible of identified syntax error of the input RDF data. 
A comprehensive error recovery can be achieve by employing  machine learning algorithms like supervised leaning methods in particular.
It would allow RDF-Doctor to learn from training data and be more effective and precise in the error correction.   

\subsection{Support other RDF Serializations }

%The current version of RDF-Doctor deals only with Turtle and N-Triple serializations. 
Supporting other RDF serializations such as JSON-LD, and RDF/XML is a crucial feature to be integrated into RDF-Doctor. 
Furthermore, the next version should able to automatically identify the RDF serializations of the input RDF and select the corresponding parser that handles the identified serialization.     

\subsection{Parsers in various Programming Languages}

RDF-Doctor is a  Java-based tool. It can be integrated as a self-contained module in another application or it can be used as a standalone software.  
Utilizing the ANTLR framework, the selected grammar should be adopted in such a way that parsers in different programming languages can be generated.%, as it has been mentioned in Chapter \ref{ch:related}.    

